\chapter{Proyecto Integrador: Sistema de Biblioteca}

\section{El Objetivo}
Vamos a construir un software completo que gestione una biblioteca. El sistema debe permitir:
\begin{itemize}
    \item Agregar libros nuevos.
    \item Listar todos los libros.
    \item Guardar la base de datos en un archivo.
    \item Cargar la base de datos al iniciar.
\end{itemize}

\section{Diseño de la Clase Libro}
Usaremos una clase simple para representar cada libro.

\begin{codebox}
\begin{lstlisting}
#include <iostream>
#include <vector>
#include <string>
#include <fstream>
#include <memory> // Para smart pointers

class Libro {
public:
    std::string titulo;
    std::string autor;
    int anio;

    Libro(std::string t, std::string a, int y) 
        : titulo(t), autor(a), anio(y) {}

    void mostrar() const {
        std::cout << "Titulo: " << titulo 
                  << " | Autor: " << autor 
                  << " (" << anio << ")\n";
    }
};
\end{lstlisting}
\end{codebox}

\section{El Gestor (La Lógica)}
Esta clase manejará la lista de libros usando un vector de punteros inteligentes.

\begin{codebox}
\begin{lstlisting}
class Biblioteca {
private:
    // Vector de punteros unicos a Libros
    std::vector<std::unique_ptr<Libro>> estanteria;

public:
    void agregarLibro() {
        std::string t, a;
        int y;
        
        std::cout << "Ingrese Titulo: ";
        std::cin.ignore(); 
        std::getline(std::cin, t);
        
        std::cout << "Ingrese Autor: ";
        std::getline(std::cin, a);
        
        std::cout << "Ingrese Anio: ";
        std::cin >> y;

        // Creamos el libro directo en el vector
        estanteria.push_back(std::make_unique<Libro>(t, a, y));
        std::cout << "Libro agregado con exito!\n";
    }

    void listarLibros() {
        if (estanteria.empty()) {
            std::cout << "La biblioteca esta vacia.\n";
            return;
        }
        
        std::cout << "\n--- Catalogo ---\n";
        for (const auto& libro : estanteria) {
            libro->mostrar();
        }
        std::cout << "----------------\n";
    }

    void guardarEnDisco() {
        std::ofstream archivo("biblioteca.txt");
        if (archivo.is_open()) {
            for (const auto& libro : estanteria) {
                // Formato CSV simple: Titulo|Autor|Anio
                archivo << libro->titulo << "|" 
                        << libro->autor << "|" 
                        << libro->anio << "\n";
            }
            std::cout << "Datos guardados en biblioteca.txt\n";
        } else {
            std::cerr << "Error al guardar archivo.\n";
        }
    }
};
\end{lstlisting}
\end{codebox}

\section{El Menú Principal}
Finalmente, conectamos todo en el main.

\begin{codebox}
\begin{lstlisting}
int main() {
    Biblioteca biblio;
    int opcion = 0;

    while (opcion != 4) {
        std::cout << "\n1. Agregar Libro\n";
        std::cout << "2. Listar Libros\n";
        std::cout << "3. Guardar y Salir\n";
        std::cout << "Seleccione: ";
        std::cin >> opcion;

        switch (opcion) {
            case 1: biblio.agregarLibro(); break;
            case 2: biblio.listarLibros(); break;
            case 3: 
                biblio.guardarEnDisco(); 
                opcion = 4; // Salir
                break;
            default: std::cout << "Opcion invalida.\n";
        }
    }
    return 0;
}
\end{lstlisting}
\end{codebox}

\section{Reto Final}
Tu misión para obtener el título de Desarrollador C++ Junior:
\begin{enumerate}
    \item Implementa la función \texttt{cargarDesdeDisco()} para que al abrir el programa, lea el archivo "biblioteca.txt" y reconstruya los objetos Libro en el vector. (Pista: usa \texttt{getline} con delimitador '|').
    \item Añade una opción para buscar un libro por título.
\end{enumerate}