\chapter{Archivos y Persistencia de Datos}

\section{Introducción a fstream}
Hasta ahora, todos los datos se pierden al cerrar el programa. Para guardarlos permanentemente (persistencia), escribimos en el disco duro.
Usamos la librería \texttt{<fstream>} (File Stream), que maneja el flujo de datos hacia archivos.

\begin{itemize}
    \item \textbf{std::ofstream:} (Output) Para escribir.
    \item \textbf{std::ifstream:} (Input) Para leer.
    \item \textbf{std::fstream:} Para lectura y escritura a la vez.
\end{itemize}

\section{Escribiendo en Archivos}
Funciona casi igual que \texttt{std::cout}.

\begin{codebox}
\begin{lstlisting}
#include <fstream>
#include <iostream>

int main() {
    // 1. Abrir el archivo (se crea si no existe)
    std::ofstream archivo("notas.txt");

    // 2. Verificar si se abrio correctamente
    if (archivo.is_open()) {
        archivo << "Lista de Tareas:\n";
        archivo << "1. Aprender C++\n";
        archivo << "2. Dominar Punteros\n";
        
        // 3. Cerrar (Buena practica, aunque el destructor lo hace)
        archivo.close();
        std::cout << "Guardado con exito.";
    } else {
        std::cerr << "Error al crear el archivo.";
    }
    return 0;
}
\end{lstlisting}
\end{codebox}

\section{Leyendo Archivos}
Leemos línea por línea o palabra por palabra.

\begin{codebox}
\begin{lstlisting}
#include <string>

int main() {
    std::ifstream lectura("notas.txt");
    std::string linea;

    if (lectura.is_open()) {
        // while(getline) lee hasta que se acabe el archivo
        while (std::getline(lectura, linea)) {
            std::cout << "Leido: " << linea << "\n";
        }
        lectura.close();
    } else {
        std::cerr << "No encuentro el archivo.";
    }
    return 0;
}
\end{lstlisting}
\end{codebox}

\section{Modos de Apertura (Flags)}
A veces no queremos borrar lo que había antes. Usamos banderas (flags) para controlar cómo se abre el archivo.

\begin{itemize}
    \item \texttt{std::ios::app} (Append): Escribe al final sin borrar.
    \item \texttt{std::ios::binary}: Modo binario (imágenes, audio).
\end{itemize}

\begin{codebox}
\begin{lstlisting}
// Abre en modo Append (anadir al final)
std::ofstream log("registro.txt", std::ios::app);
log << "Nuevo evento registrado.\n";
\end{lstlisting}
\end{codebox}

\newpage
\section{Evaluación Profesional}

\subsection{Preguntas de Ingeniería}
\begin{enumerate}
    \item \textbf{Buffer:} ¿Qué significa que la escritura en disco tiene "buffer"? ¿Por qué el texto no aparece inmediatamente en el archivo?
    \item \textbf{Rutas:} ¿Cuál es la diferencia entre una ruta relativa ("datos.txt") y una absoluta ("C:/Usuarios/...")?
    \item \textbf{Binario vs Texto:} ¿Por qué no deberías abrir una imagen JPG en modo texto predeterminado?
\end{enumerate}

\subsection{Retos}
\begin{enumerate}
    \item **Logger:** Crea una función `log(mensaje)` que escriba errores en un archivo sin borrar los anteriores.
    \item **Configuración:** Lee un archivo `config.txt` que tenga `puerto=8080` y extrae el número.
    \item **Copia:** Haz un programa que copie un archivo origen a uno destino byte por byte.
\end{enumerate}