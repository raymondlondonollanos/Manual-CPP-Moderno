\chapter{Configuración del Entorno}

\section{El Compilador}
El compilador es el programa encargado de traducir tu código C++ (humanamente legible) a código máquina (binario).

Los tres más populares son:
\begin{itemize}
    \item \textbf{GCC (GNU Compiler Collection):} El estándar en Linux.
    \item \textbf{MSVC (Microsoft Visual C++):} El estándar en Windows (Visual Studio).
    \item \textbf{Clang:} Muy usado en Mac y Android.
\end{itemize}

\section{El IDE (Entorno de Desarrollo)}
Aunque puedes usar el Bloc de Notas, se recomienda un IDE profesional.

\subsection{Visual Studio Code (Recomendado)}
Es ligero, gratuito y tiene excelentes extensiones para C++.
Pasos básicos:
\begin{enumerate}
    \item Descargar VS Code desde la web oficial.
    \item Instalar la extensión "C/C++" de Microsoft.
    \item Instalar un compilador (MinGW en Windows o g++ en Linux).
\end{enumerate}

\section{Proceso de Compilación}
El proceso tiene 3 etapas:
\begin{enumerate}
    \item \textbf{Preprocesado:} Maneja los \#include y comentarios.
    \item \textbf{Compilación:} Crea archivos objeto (.o / .obj).
    \item \textbf{Enlazado (Linker):} Junta todo en un ejecutable (.exe).
\end{enumerate}