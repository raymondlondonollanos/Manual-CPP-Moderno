\chapter{Control de Flujo: Decisiones y Bucles}

\section{Toma de Decisiones (Condicionales)}
El control de flujo permite que nuestro programa tome caminos diferentes.

\subsection{La sentencia if / else}
Evalúa una condición booleana (Verdadero/Falso).

\begin{codebox}
\begin{lstlisting}
int edad = 18;

if (edad >= 18) {
    std::cout << "Acceso permitido.";
} else if (edad > 12) {
    std::cout << "Necesitas supervision.";
} else {
    std::cout << "Acceso denegado.";
}
\end{lstlisting}
\end{codebox}

\section{La sentencia switch}
Útil para comparar una variable contra múltiples valores fijos.

\begin{codebox}
\begin{lstlisting}
int opcion = 1;
switch (opcion) {
    case 1: std::cout << "Opcion 1"; break;
    case 2: std::cout << "Opcion 2"; break;
    default: std::cout << "Nada"; break;
}
\end{lstlisting}
\end{codebox}

\section{Bucles (Loops)}

\subsection{Bucle While}
Repite mientras la condición sea verdadera.

\begin{codebox}
\begin{lstlisting}
int i = 0;
while (i < 5) {
    std::cout << i << " ";
    i++;
}
\end{lstlisting}
\end{codebox}

\subsection{Bucle For}
El clásico para contar.

\begin{codebox}
\begin{lstlisting}
for (int i = 0; i < 5; ++i) {
    std::cout << "Iteracion: " << i << std::endl;
}
\end{lstlisting}
\end{codebox}