\chapter{Clases y Objetos (POO)}

\section{Introducción a la POO}
La Programación Orientada a Objetos (POO) es un paradigma que nos permite organizar el código modelando cosas de la vida real.

\textbf{La Analogía del Plano y la Casa:}
\begin{itemize}
    \item \textbf{Clase (Class):} Es el \textit{plano} o molde. Define cómo debe ser algo (propiedades y acciones), pero no existe físicamente.
    \item \textbf{Objeto (Object):} Es la \textit{casa} construida usando el plano. Es una instancia real que ocupa memoria.
\end{itemize}

\section{Creando una Clase}
En C++, usamos la palabra clave \texttt{class}. Una clase tiene **Atributos** (variables) y **Métodos** (funciones).

\begin{codebox}
\begin{lstlisting}
class Jugador {
public:
    // Atributos (Estado)
    std::string nombre;
    int salud;
    int nivel;

    // Metodos (Comportamiento)
    void saludar() {
        std::cout << "Hola, soy " << nombre << "\n";
    }
}; // IMPORTANTE: Punto y coma al final
\end{lstlisting}
\end{codebox}

\section{Instanciando Objetos}
Una vez tenemos el plano, podemos crear objetos en el \texttt{main}.

\begin{codebox}
\begin{lstlisting}
int main() {
    // Crear objeto (Instanciar)
    Jugador p1;
    p1.nombre = "Mario";
    p1.salud = 100;
    
    // Usar sus metodos
    p1.saludar(); // Imprime: Hola, soy Mario
    return 0;
}
\end{lstlisting}
\end{codebox}

\section{Control de Acceso (Encapsulamiento)}
Es la práctica de ocultar los datos internos para protegerlos.
\begin{itemize}
    \item \textbf{public:} Accesible desde cualquier parte.
    \item \textbf{private:} Accesible SOLO desde dentro de la clase.
\end{itemize}

\begin{codebox}
\begin{lstlisting}
class CuentaBancaria {
private:
    double saldo; // Nadie puede tocar esto directamente

public:
    void depositar(double cantidad) {
        if (cantidad > 0) {
            saldo += cantidad; // Solo modificamos si es valido
        }
    }
};
\end{lstlisting}
\end{codebox}

\section{Constructores}
El constructor es un método especial que se ejecuta automáticamente cuando nace el objeto. Se usa para inicializar valores. Se llama igual que la clase.

\begin{codebox}
\begin{lstlisting}
class Coche {
public:
    std::string marca;
    int year;

    // Constructor
    Coche(std::string m, int y) {
        marca = m;
        year = y;
        std::cout << "Coche creado!\n";
    }
};

int main() {
    // Se llama al constructor aqui
    Coche miAuto("Toyota", 2022); 
}
\end{lstlisting}
\end{codebox}

\newpage
\section{Evaluación y Práctica Profesional}

\subsection{Conceptos de Ingeniería (5 Preguntas)}
\begin{enumerate}
    \item Explica con tus palabras la diferencia entre Clase y Objeto.
    \item ¿Por qué es buena práctica declarar los atributos como \texttt{private} (Encapsulamiento)?
    \item ¿Qué es el puntero \texttt{this} dentro de una clase?
    \item ¿Qué características especiales tiene un Constructor (nombre y tipo de retorno)?
    \item Diferencia técnica entre \texttt{struct} y \texttt{class} en C++ (Pista: visibilidad por defecto).
\end{enumerate}

\subsection{Retos de Desarrollo (10 Ejercicios)}
\begin{enumerate}
    \item Crea una clase \texttt{Gato} con atributos nombre y edad, y método \texttt{maullar()}.
    \item Implementa una clase \texttt{Rectangulo} con constructor que reciba ancho y alto, y un método para calcular el área.
    \item **Validación:** Crea una clase \texttt{Persona} con edad privada. El método \texttt{setEdad} debe rechazar números negativos.
    \item Crea un vector de objetos \texttt{Estudiante}. Agrega 3 estudiantes y recórrelo.
    \item Implementa una clase \texttt{Libro} que imprima "Titulo: X, Autor: Y".
    \item **Sobrecarga de Constructores:** Crea una clase con dos constructores distintos (uno vacío y otro con parámetros).
    \item **RPG:** Crea una clase \texttt{Enemigo} con vida. Crea un método \texttt{recibirDano(int)} que reste vida y avise si murió.
    \item Crea una clase \texttt{Calculadora} con métodos sumar, restar, multiplicar y dividir.
    \item Simula un cajero automático simple usando una clase \texttt{Cuenta} con métodos retirar e ingresar.
    \item **Destructor:** Implementa un destructor (\texttt{\~Clase}) que imprima un mensaje cuando el objeto sea destruido.
\end{enumerate}

\subsection{Revisión de Código (Auditoría)}
\begin{codebox}
\begin{lstlisting}
// Caso 1: Acceso Privado
class Caja {
    int secreto; // Private por defecto
public:
    void abrir() {}
};
int main() {
    Caja c;
    // c.secreto = 10; // Error: Inaccesible
}

// Caso 2: Constructor Mal Formado
class Auto {
public:
    // void Auto() {} // Error: Constructor no lleva void
    Auto() {} // Correcto
};

// Caso 3: Instancia Incorrecta
int main() {
    // Robot r(); // Error: Esto declara una funcion, no un objeto
    // Robot r;   // Correcto
}
\end{lstlisting}
\end{codebox}