\chapter{La STL: Algoritmos y Lambdas}

\section{La Filosofía de la STL}
No reinventes la rueda. Antes de escribir un bucle para buscar u ordenar, revisa la librería \texttt{<algorithm>}.

\section{Algoritmos Comunes}
\begin{codebox}
\begin{lstlisting}
#include <algorithm>
#include <vector>
#include <iostream>

int main() {
    std::vector<int> nums = {5, 2, 9, 1};

    // Ordenar (Sort)
    std::sort(nums.begin(), nums.end()); 

    // Buscar (Find)
    auto it = std::find(nums.begin(), nums.end(), 9);
    
    if (it != nums.end()) {
        std::cout << "Encontrado: " << *it;
    }
    return 0;
}
\end{lstlisting}
\end{codebox}

\section{Lambdas (Funciones Anónimas)}
Son funciones pequeñas escritas "al vuelo".
\texttt{[](params) \{ cuerpo \}}

\begin{codebox}
\begin{lstlisting}
std::vector<int> datos = {1, 2, 3};
int multiplicador = 10;

// [=] captura variables externas por copia
std::for_each(datos.begin(), datos.end(), [=](int n) {
    std::cout << n * multiplicador << " ";
});
\end{lstlisting}
\end{codebox}

\newpage
\section{Evaluación Profesional}

\subsection{Ejercicios}
\begin{enumerate}
    \item **Sort Personalizado:** Ordena un vector de mayor a menor usando una lambda.
    \item **Filtrado:** Usa \texttt{std::copy\_if} para copiar solo números pares.
    \item **Búsqueda Avanzada:** Usa \texttt{std::find\_if} para encontrar el primer número mayor a 100.
    \item **Transformación:** Usa \texttt{std::transform} para convertir strings a mayúsculas.
    \item **Contador:** Cuenta cuántos negativos hay con \texttt{std::count\_if}.
\end{enumerate}

\subsection{Revisión de Código}
\begin{codebox}
\begin{lstlisting}
std::vector<int> v = {1, 2, 3};
// std::sort(v.begin(), v.begin()); 
// Error: Rango vacio o invalido.

auto it = std::find(v.begin(), v.end(), 5);
// std::cout << *it; 
// Error: Si no lo encuentra, it es v.end().
// Dereferenciar end() es comportamiento indefinido.
\end{lstlisting}
\end{codebox}