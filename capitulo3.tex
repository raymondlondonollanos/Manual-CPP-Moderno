\chapter{Tu Primer Programa}

\section{Hola Mundo en C++}
Vamos a escribir el programa más famoso de la historia.

\begin{codebox}
\begin{lstlisting}
#include <iostream>

int main() {
    std::cout << "Hola Mundo!" << std::endl;
    return 0;
}
\end{lstlisting}
\end{codebox}

\section{Análisis del Código}

\subsection{La cabecera include}
\texttt{\#include <iostream>} le dice al compilador que queremos usar la librería de entrada y salida (Input Output Stream). Sin esto, no podemos usar \texttt{cout}.

\subsection{La función main}
Todo programa en C++ comienza en la función \texttt{main}. Es el punto de entrada.
\begin{itemize}
    \item Devuelve un entero (\texttt{int}).
    \item Si devuelve 0, significa "Exito".
    \item Si devuelve otro número, significa "Error".
\end{itemize}

\subsection{std::cout}
Significa "Character Output". Se usa con el operador de inserción \texttt{<{}<} para enviar texto a la consola.