\chapter{Variables y Tipos de Datos}

\section{¿Qué es una variable?}
Una variable es un espacio en la memoria RAM donde guardamos un dato. En C++, debemos decir qué \textbf{tipo} de dato vamos a guardar.

\section{Tipos Primitivos Básicos}

\begin{itemize}
    \item \textbf{int:} Números enteros (ej: 5, -10, 400).
    \item \textbf{double:} Números con decimales (ej: 3.14, 9.99).
    \item \textbf{char:} Un solo carácter (ej: 'a', 'Z').
    \item \textbf{bool:} Verdadero o Falso (true/false).
\end{itemize}

\section{Declaración e Inicialización}
En C++ Moderno (C++11 en adelante), recomendamos usar llaves para inicializar.

\begin{codebox}
\begin{lstlisting}
int vidas{3};           // Correcto y moderno
double precio = 99.90;  // Estilo clasico
char letra{'A'};        
bool esDeDia{true};
\end{lstlisting}
\end{codebox}

\section{Entrada de Datos (cin)}
Para leer lo que el usuario escribe, usamos \texttt{std::cin} con el operador de extracción \texttt{>{}>}.

\begin{codebox}
\begin{lstlisting}
int edad;
std::cout << "Introduce tu edad: ";
std::cin >> edad;
std::cout << "Tienes " << edad << " anios.";
\end{lstlisting}
\end{codebox}