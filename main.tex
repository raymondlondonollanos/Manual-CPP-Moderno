\documentclass[12pt, a4paper]{book}
\usepackage[spanish]{babel}
\usepackage[utf8]{inputenc}
\usepackage[T1]{fontenc}
\usepackage{geometry}
\usepackage{graphicx}
\usepackage{xcolor}
\usepackage{pagecolor}
\usepackage{titlesec}
\usepackage[most]{tcolorbox}
\usepackage{listings}

% --- PAQUETES PARA EL MAPA (ROADMAP) ---
\usepackage{tikz}
\usetikzlibrary{shadows, arrows.meta, positioning}

% --- COLORES ---
\definecolor{FondoCrema}{RGB}{253, 250, 245}
\definecolor{FondoCodigo}{RGB}{245, 245, 245}
\definecolor{BordeCodigo}{RGB}{220, 220, 220}
\definecolor{AzulTech}{RGB}{0, 102, 204} 
\definecolor{lcppBlue}{HTML}{0000FF}
\definecolor{lcppGreen}{HTML}{008000}
\definecolor{lcppString}{HTML}{A31515}

% --- CONFIGURACIÓN DE HIPERVÍNCULOS (QUITAR CUADROS ROJOS) ---
\usepackage{hyperref}
\hypersetup{
    colorlinks=true,       % true = colorear texto, false = hacer cuadros
    linkcolor=black,       % Color del Índice y referencias internas (Negro = Elegante)
    urlcolor=AzulTech,     % Color de URLs externas (Azul)
    citecolor=black,
    pdftitle={Manual Profesional C++},
    pdfauthor={Raymond Emilio Londoño Llanos}
}

% --- CONFIGURACIÓN DE CÓDIGO ---
\lstdefinestyle{learncppStyle}{
    language=C++,
    backgroundcolor=\color{FondoCodigo},
    basicstyle=\ttfamily\small\color{black},
    keywordstyle=\color{lcppBlue}\bfseries,
    commentstyle=\color{lcppGreen},
    stringstyle=\color{lcppString},
    numbers=left,
    numberstyle=\tiny\color{gray},
    stepnumber=1,
    numbersep=10pt,
    tabsize=4,
    breaklines=true,
    frame=none,
    inputencoding=utf8,
    extendedchars=true,
    literate={á}{{\'a}}1 {é}{{\'e}}1 {í}{{\'i}}1 {ó}{{\'o}}1 {ú}{{\'u}}1 {ñ}{{\~n}}1
}
\lstset{style=learncppStyle}

% Caja contenedora bonita
\newtcolorbox{codebox}[1][]{
    colback=FondoCodigo,
    colframe=BordeCodigo,
    boxrule=0.5pt,
    arc=4pt,
    left=5pt, right=5pt, top=5pt, bottom=5pt,
    enhanced,
    #1
}

% --- FORMATO DE PÁGINA ---
\geometry{top=3cm, bottom=3cm, left=3cm, right=3cm}
\pagecolor{FondoCrema}

% Formato de Títulos de Capítulo
\titleformat{\chapter}[display]
  {\normalfont\huge\bfseries\color{black}}{\chaptertitlename\ \thechapter}{20pt}{\Huge\color{AzulTech}}

\begin{document}

% =========================================
%             PORTADA PERSONALIZADA
% =========================================
\begin{titlepage}
    \centering
    \vspace*{2cm}
    
    % Título Principal
    \rule{\linewidth}{0.5mm} \\[0.4cm]
    { \huge \bfseries MANUAL PROFESIONAL DE C++ MODERNO \\[0.2cm] }
    { \Large De Principiante a Ingeniero de Software }
    \rule{\linewidth}{0.5mm} \\[1.5cm]
    
    \vspace{2cm}
    
    % Decoración simple
    \begin{tikzpicture}
        \draw[AzulTech, line width=2pt] (0,0) -- (6,0);
    \end{tikzpicture}
    
    \vfill
    
    % Autor
    \textbf{\Large Autor:} \\
    {\Large Raymond Emilio Londoño Llanos}
    
    \vspace{0.5cm}
    
    {\large Ingeniero de Software}
    
    \vfill
    
    {\large Colombia, 2026}
    
    \vspace*{1cm}
\end{titlepage}

% =========================================
%             AGRADECIMIENTOS
% =========================================
\chapter*{Agradecimientos}
Este libro es el resultado de la perseverancia, la curiosidad y la pasión por el código.

Agradezco a la comunidad de C++ por construir herramientas tan poderosas, y a todos los que comparten conocimiento libremente.

Gracias a ti, lector, por aceptar el reto de dominar uno de los lenguajes más importantes de la historia de la computación.

\tableofcontents

% =========================================
%             EL MAPA DEL TESORO
% =========================================
\include{roadmap}

% =========================================
%             CONTENIDO TÉCNICO
% =========================================

\chapter{Introducción a C++}

\section{¿Qué es C++? ¿Por qué usar C++ Moderno?}

\subsection{¿Qué es C++?}
C++ es un lenguaje de programación de propósito general que combina la eficiencia y las capacidades de bajo nivel de C con características avanzadas como la programación orientada a objetos (POO), programación genérica y programación funcional.

Fue creado por \textbf{Bjarne Stroustrup} en 1979 en los laboratorios Bell como una extensión del lenguaje C. Su objetivo era añadir características de POO a C sin sacrificar su rendimiento ni flexibilidad.

\begin{itemize}
    \item \textbf{Eficiencia:} Permite un control granular sobre los recursos del sistema.
    \item \textbf{Flexibilidad:} Soporta múltiples paradigmas de programación.
    \item \textbf{Compatibilidad:} Mantiene compatibilidad con C y estándares antiguos.
\end{itemize}

\subsection{¿Por qué usar C++ Moderno?}
El término "C++ Moderno" se refiere a los estándares introducidos a partir de C++11 (C++14, C++17, C++20, C++23). Estos estándares han transformado el lenguaje haciéndolo más seguro, expresivo y fácil de usar.

\subsubsection{Mejora en Legibilidad y Expresividad}
Una de las mejoras más notables es la palabra clave \texttt{auto}, que deduce automáticamente el tipo de una variable:

\begin{lstlisting}[language=C++, caption=Ejemplo de auto en C++]
auto x = 42; // x se deduce como int
auto nombre = "C++"; // nombre se deduce como const char*
\end{lstlisting}

\subsubsection{Seguridad y Gestión de Memoria}
El C++ Moderno introduce los \textbf{Punteros Inteligentes} (Smart Pointers), que gestionan la memoria automáticamente:

\begin{lstlisting}[language=C++, caption=Uso de unique\_ptr]
#include <memory>
std::unique_ptr<int> ptr = std::make_unique<int>(42);
// No es necesario borrar ptr manualmente; se limpia solo.
\end{lstlisting}

\section{Historia de C++ y la Evolución de los Estándares}
\subsection{Desde C++11 hasta C++23}
\begin{itemize}
    \item \textbf{C++11:} El gran salto. Introdujo \texttt{auto}, lambdas y punteros inteligentes.
    \item \textbf{C++14 y C++17:} Refinamientos, algoritmos paralelos y \texttt{std::filesystem}.
    \item \textbf{C++20:} Una revolución con "Concepts", "Coroutines" y "Modules".
\end{itemize}
\chapter{Configuración del Entorno}

\section{El Compilador}
El compilador es el programa encargado de traducir tu código C++ (humanamente legible) a código máquina (binario).

Los tres más populares son:
\begin{itemize}
    \item \textbf{GCC (GNU Compiler Collection):} El estándar en Linux.
    \item \textbf{MSVC (Microsoft Visual C++):} El estándar en Windows (Visual Studio).
    \item \textbf{Clang:} Muy usado en Mac y Android.
\end{itemize}

\section{El IDE (Entorno de Desarrollo)}
Aunque puedes usar el Bloc de Notas, se recomienda un IDE profesional.

\subsection{Visual Studio Code (Recomendado)}
Es ligero, gratuito y tiene excelentes extensiones para C++.
Pasos básicos:
\begin{enumerate}
    \item Descargar VS Code desde la web oficial.
    \item Instalar la extensión "C/C++" de Microsoft.
    \item Instalar un compilador (MinGW en Windows o g++ en Linux).
\end{enumerate}

\section{Proceso de Compilación}
El proceso tiene 3 etapas:
\begin{enumerate}
    \item \textbf{Preprocesado:} Maneja los \#include y comentarios.
    \item \textbf{Compilación:} Crea archivos objeto (.o / .obj).
    \item \textbf{Enlazado (Linker):} Junta todo en un ejecutable (.exe).
\end{enumerate}
\chapter{Tu Primer Programa}

\section{Hola Mundo en C++}
Vamos a escribir el programa más famoso de la historia.

\begin{codebox}
\begin{lstlisting}
#include <iostream>

int main() {
    std::cout << "Hola Mundo!" << std::endl;
    return 0;
}
\end{lstlisting}
\end{codebox}

\section{Análisis del Código}

\subsection{La cabecera include}
\texttt{\#include <iostream>} le dice al compilador que queremos usar la librería de entrada y salida (Input Output Stream). Sin esto, no podemos usar \texttt{cout}.

\subsection{La función main}
Todo programa en C++ comienza en la función \texttt{main}. Es el punto de entrada.
\begin{itemize}
    \item Devuelve un entero (\texttt{int}).
    \item Si devuelve 0, significa "Exito".
    \item Si devuelve otro número, significa "Error".
\end{itemize}

\subsection{std::cout}
Significa "Character Output". Se usa con el operador de inserción \texttt{<{}<} para enviar texto a la consola.
\chapter{Variables y Tipos de Datos}

\section{¿Qué es una variable?}
Una variable es un espacio en la memoria RAM donde guardamos un dato. En C++, debemos decir qué \textbf{tipo} de dato vamos a guardar.

\section{Tipos Primitivos Básicos}

\begin{itemize}
    \item \textbf{int:} Números enteros (ej: 5, -10, 400).
    \item \textbf{double:} Números con decimales (ej: 3.14, 9.99).
    \item \textbf{char:} Un solo carácter (ej: 'a', 'Z').
    \item \textbf{bool:} Verdadero o Falso (true/false).
\end{itemize}

\section{Declaración e Inicialización}
En C++ Moderno (C++11 en adelante), recomendamos usar llaves para inicializar.

\begin{codebox}
\begin{lstlisting}
int vidas{3};           // Correcto y moderno
double precio = 99.90;  // Estilo clasico
char letra{'A'};        
bool esDeDia{true};
\end{lstlisting}
\end{codebox}

\section{Entrada de Datos (cin)}
Para leer lo que el usuario escribe, usamos \texttt{std::cin} con el operador de extracción \texttt{>{}>}.

\begin{codebox}
\begin{lstlisting}
int edad;
std::cout << "Introduce tu edad: ";
std::cin >> edad;
std::cout << "Tienes " << edad << " anios.";
\end{lstlisting}
\end{codebox}
\chapter{Control de Flujo: Decisiones y Bucles}

\section{Toma de Decisiones (Condicionales)}
El control de flujo permite que nuestro programa tome caminos diferentes.

\subsection{La sentencia if / else}
Evalúa una condición booleana (Verdadero/Falso).

\begin{codebox}
\begin{lstlisting}
int edad = 18;

if (edad >= 18) {
    std::cout << "Acceso permitido.";
} else if (edad > 12) {
    std::cout << "Necesitas supervision.";
} else {
    std::cout << "Acceso denegado.";
}
\end{lstlisting}
\end{codebox}

\section{La sentencia switch}
Útil para comparar una variable contra múltiples valores fijos.

\begin{codebox}
\begin{lstlisting}
int opcion = 1;
switch (opcion) {
    case 1: std::cout << "Opcion 1"; break;
    case 2: std::cout << "Opcion 2"; break;
    default: std::cout << "Nada"; break;
}
\end{lstlisting}
\end{codebox}

\section{Bucles (Loops)}

\subsection{Bucle While}
Repite mientras la condición sea verdadera.

\begin{codebox}
\begin{lstlisting}
int i = 0;
while (i < 5) {
    std::cout << i << " ";
    i++;
}
\end{lstlisting}
\end{codebox}

\subsection{Bucle For}
El clásico para contar.

\begin{codebox}
\begin{lstlisting}
for (int i = 0; i < 5; ++i) {
    std::cout << "Iteracion: " << i << std::endl;
}
\end{lstlisting}
\end{codebox}
\chapter{Funciones y Modularidad}

\section{Conceptos Básicos}
Las funciones nos permiten reutilizar código y dividir problemas complejos.

\begin{codebox}
\begin{lstlisting}
// Tipo Nombre(Parametros) { Cuerpo }

int sumar(int a, int b) {
    return a + b;
}

void saludar() {
    std::cout << "Hola desde una funcion!";
}
\end{lstlisting}
\end{codebox}

\section{Paso de Parámetros}

\subsection{Paso por Valor}
Se crea una copia. El original no cambia.

\begin{codebox}
\begin{lstlisting}
void intento(int x) { x = 100; } // No afecta al original
\end{lstlisting}
\end{codebox}

\subsection{Paso por Referencia (\&)}
Se accede a la variable original.

\begin{codebox}
\begin{lstlisting}
void cambiar(int& x) { x = 100; } // Si afecta al original
\end{lstlisting}
\end{codebox}
\chapter{Arrays y Vectores: Manejo de Colecciones}

\section{Introducción}
Hasta ahora, si queriamos guardar las notas de 20 estudiantes, teniamos que crear 20 variables distintas. Eso es inmanejable.
En C++, usamos **contenedores** para almacenar multiples valores bajo un mismo nombre.

\section{std::vector (La Navaja Suiza)}
El \texttt{std::vector} es un array dinamico. A diferencia de los arrays antiguos de C, el vector conoce su propio tamaño, puede crecer y encogerse, y gestiona la memoria por ti. Es la coleccion por defecto en C++ Moderno.

Para usarlo: \texttt{\#include <vector>}

\subsection{Declaración e Inicialización}

\begin{codebox}
\begin{lstlisting}
#include <vector>
#include <iostream>

int main() {
    // 1. Vector vacio de enteros
    std::vector<int> numeros;

    // 2. Vector inicializado con valores
    std::vector<int> primos = {2, 3, 5, 7, 11};

    // 3. Vector de tamano 10, todo inicializado a 0
    std::vector<int> ceros(10, 0);
    
    return 0;
}
\end{lstlisting}
\end{codebox}

\subsection{Operaciones Básicas}
El vector tiene metodos muy utiles.

\begin{codebox}
\begin{lstlisting}
std::vector<std::string> frutas;

// push_back: Agrega elementos al final de la lista
frutas.push_back("Manzana");
frutas.push_back("Banana");
frutas.push_back("Uva");

// Acceso: Se usa el indice comenzando en 0
std::cout << "Primera fruta: " << frutas[0]; // Manzana

// size: Devuelve la cantidad de elementos
std::cout << "Total: " << frutas.size();

// pop_back: Elimina el ultimo elemento
frutas.pop_back(); // Adios "Uva"
\end{lstlisting}
\end{codebox}

\section{std::array (La Alternativa Fija)}
Introducido en C++11, \texttt{std::array} reemplaza a los arrays estilo C.
Su tamaño es **fijo** y debe conocerse al compilar.

\begin{codebox}
\begin{lstlisting}
#include <array>

// Sintaxis: std::array<Tipo, Tamano> nombre;
std::array<int, 3> coordenadas = {10, 20, 30};

std::cout << coordenadas[1]; // 20

// coordenadas.push_back(40); // ERROR: No puede cambiar de tamano
\end{lstlisting}
\end{codebox}

\section{Iterando sobre Colecciones}
La mejor forma de recorrer un vector o array es con el **bucle for basado en rangos**.

\begin{codebox}
\begin{lstlisting}
std::vector<int> notas = {85, 90, 78};

// Estilo moderno y seguro:
for (int nota : notas) {
    std::cout << nota << " ";
}
\end{lstlisting}
\end{codebox}

\newpage
\section{Evaluación y Práctica Profesional}

\subsection{Conceptos de Ingeniería (5 Preguntas)}
\begin{enumerate}
    \item \textbf{Vector vs Array:} ¿En qué escenario específico preferirías usar un \texttt{std::array} sobre un \texttt{std::vector}?
    \item \textbf{Seguridad:} ¿Qué sucede si intentas acceder al índice 10 de un vector que solo tiene 5 elementos?
    \item \textbf{Complejidad:} ¿Cuál es el costo computacional de insertar un elemento al final de un vector (\texttt{push\_back})?
    \item \textbf{Capacidad:} Explica la diferencia entre \texttt{.size()} y \texttt{.capacity()}.
    \item \textbf{Base-Zero:} ¿Por qué los índices en C++ empiezan en 0?
\end{enumerate}

\subsection{Retos de Desarrollo (10 Ejercicios)}
\begin{enumerate}
    \item Declara un \texttt{vector} vacío y llénalo con los números del 1 al 100.
    \item Crea un vector de \texttt{string} con nombres de ciudades.
    \item **Suma:** Crea una función que reciba un \texttt{const std::vector<int>\&} y devuelva la suma.
    \item **Máximo:** Encuentra el número más grande dentro de un vector.
    \item **Matriz:** Crea un vector de vectores para representar una matriz de 3x3.
    \item **Filtrado:** Crea un vector nuevo que contenga solo los números pares del original.
    \item **Inversión:** Imprime un vector desde el último elemento hasta el primero.
    \item Usa \texttt{std::array} para representar un punto 3D (x, y, z).
    \item **Validación:** Escribe un programa que pida notas (0-10) hasta que el usuario escriba -1.
    \item **Copia:** Copia el contenido de un vector A a un vector B manualmente.
\end{enumerate}

\subsection{Revisión de Código}
\begin{codebox}
\begin{lstlisting}
std::vector<int> v = {1, 2, 3};
// std::cout << v[5]; // ERROR: Acceso fuera de limites

std::array<int, 3> a = {1, 2, 3};
// a.push_back(4); // ERROR: array no tiene push_back
\end{lstlisting}
\end{codebox}
\chapter{Clases y Objetos (POO)}

\section{Introducción a la POO}
La Programación Orientada a Objetos (POO) es un paradigma que nos permite organizar el código modelando cosas de la vida real.

\textbf{La Analogía del Plano y la Casa:}
\begin{itemize}
    \item \textbf{Clase (Class):} Es el \textit{plano} o molde. Define cómo debe ser algo (propiedades y acciones), pero no existe físicamente.
    \item \textbf{Objeto (Object):} Es la \textit{casa} construida usando el plano. Es una instancia real que ocupa memoria.
\end{itemize}

\section{Creando una Clase}
En C++, usamos la palabra clave \texttt{class}. Una clase tiene **Atributos** (variables) y **Métodos** (funciones).

\begin{codebox}
\begin{lstlisting}
class Jugador {
public:
    // Atributos (Estado)
    std::string nombre;
    int salud;
    int nivel;

    // Metodos (Comportamiento)
    void saludar() {
        std::cout << "Hola, soy " << nombre << "\n";
    }
}; // IMPORTANTE: Punto y coma al final
\end{lstlisting}
\end{codebox}

\section{Instanciando Objetos}
Una vez tenemos el plano, podemos crear objetos en el \texttt{main}.

\begin{codebox}
\begin{lstlisting}
int main() {
    // Crear objeto (Instanciar)
    Jugador p1;
    p1.nombre = "Mario";
    p1.salud = 100;
    
    // Usar sus metodos
    p1.saludar(); // Imprime: Hola, soy Mario
    return 0;
}
\end{lstlisting}
\end{codebox}

\section{Control de Acceso (Encapsulamiento)}
Es la práctica de ocultar los datos internos para protegerlos.
\begin{itemize}
    \item \textbf{public:} Accesible desde cualquier parte.
    \item \textbf{private:} Accesible SOLO desde dentro de la clase.
\end{itemize}

\begin{codebox}
\begin{lstlisting}
class CuentaBancaria {
private:
    double saldo; // Nadie puede tocar esto directamente

public:
    void depositar(double cantidad) {
        if (cantidad > 0) {
            saldo += cantidad; // Solo modificamos si es valido
        }
    }
};
\end{lstlisting}
\end{codebox}

\section{Constructores}
El constructor es un método especial que se ejecuta automáticamente cuando nace el objeto. Se usa para inicializar valores. Se llama igual que la clase.

\begin{codebox}
\begin{lstlisting}
class Coche {
public:
    std::string marca;
    int year;

    // Constructor
    Coche(std::string m, int y) {
        marca = m;
        year = y;
        std::cout << "Coche creado!\n";
    }
};

int main() {
    // Se llama al constructor aqui
    Coche miAuto("Toyota", 2022); 
}
\end{lstlisting}
\end{codebox}

\newpage
\section{Evaluación y Práctica Profesional}

\subsection{Conceptos de Ingeniería (5 Preguntas)}
\begin{enumerate}
    \item Explica con tus palabras la diferencia entre Clase y Objeto.
    \item ¿Por qué es buena práctica declarar los atributos como \texttt{private} (Encapsulamiento)?
    \item ¿Qué es el puntero \texttt{this} dentro de una clase?
    \item ¿Qué características especiales tiene un Constructor (nombre y tipo de retorno)?
    \item Diferencia técnica entre \texttt{struct} y \texttt{class} en C++ (Pista: visibilidad por defecto).
\end{enumerate}

\subsection{Retos de Desarrollo (10 Ejercicios)}
\begin{enumerate}
    \item Crea una clase \texttt{Gato} con atributos nombre y edad, y método \texttt{maullar()}.
    \item Implementa una clase \texttt{Rectangulo} con constructor que reciba ancho y alto, y un método para calcular el área.
    \item **Validación:** Crea una clase \texttt{Persona} con edad privada. El método \texttt{setEdad} debe rechazar números negativos.
    \item Crea un vector de objetos \texttt{Estudiante}. Agrega 3 estudiantes y recórrelo.
    \item Implementa una clase \texttt{Libro} que imprima "Titulo: X, Autor: Y".
    \item **Sobrecarga de Constructores:** Crea una clase con dos constructores distintos (uno vacío y otro con parámetros).
    \item **RPG:** Crea una clase \texttt{Enemigo} con vida. Crea un método \texttt{recibirDano(int)} que reste vida y avise si murió.
    \item Crea una clase \texttt{Calculadora} con métodos sumar, restar, multiplicar y dividir.
    \item Simula un cajero automático simple usando una clase \texttt{Cuenta} con métodos retirar e ingresar.
    \item **Destructor:** Implementa un destructor (\texttt{\~Clase}) que imprima un mensaje cuando el objeto sea destruido.
\end{enumerate}

\subsection{Revisión de Código (Auditoría)}
\begin{codebox}
\begin{lstlisting}
// Caso 1: Acceso Privado
class Caja {
    int secreto; // Private por defecto
public:
    void abrir() {}
};
int main() {
    Caja c;
    // c.secreto = 10; // Error: Inaccesible
}

// Caso 2: Constructor Mal Formado
class Auto {
public:
    // void Auto() {} // Error: Constructor no lleva void
    Auto() {} // Correcto
};

// Caso 3: Instancia Incorrecta
int main() {
    // Robot r(); // Error: Esto declara una funcion, no un objeto
    // Robot r;   // Correcto
}
\end{lstlisting}
\end{codebox}
\chapter{Punteros y Gestión de Memoria}

\section{La Verdad sobre la Memoria}
Toda variable vive en una dirección de memoria. Un **puntero** es simplemente una variable que guarda esa dirección.

\subsection{Punteros Crudos (Raw Pointers)}
Son la forma antigua de trabajar. Potentes pero peligrosos.

\begin{codebox}
\begin{lstlisting}
int x = 10;
int* ptr = &x; // & obtiene la direccion

std::cout << ptr;  // Imprime direccion
std::cout << *ptr; // Imprime valor (10)
\end{lstlisting}
\end{codebox}

\section{Memoria Dinámica: Heap vs Stack}
\begin{itemize}
    \item \textbf{Stack (Pila):} Memoria automática. Las variables se borran al cerrar la llave \texttt{\}}.
    \item \textbf{Heap (Montón):} Memoria manual. Las variables viven hasta que tú las borras.
\end{itemize}

\section{C++ Moderno: Smart Pointers}
En el mundo profesional, usamos punteros inteligentes que se borran solos (RAII).

\subsection{1. std::unique\_ptr}
Propiedad exclusiva. Solo un dueño a la vez. Cuando el puntero sale de ámbito, la memoria se libera.

\begin{codebox}
\begin{lstlisting}
#include <memory>

void nivel() {
    // Fabrica un unique_ptr
    std::unique_ptr<int> boss = std::make_unique<int>(500);
    
    // Al cerrar la funcion, 'boss' se destruye solo.
}
\end{lstlisting}
\end{codebox}

\subsection{2. std::shared\_ptr}
Propiedad compartida. La memoria solo se libera cuando el **último** puntero deja de apuntar al objeto.

\newpage
\section{Evaluación y Práctica Profesional}

\subsection{Conceptos de Ingeniería (5 Preguntas)}
\begin{enumerate}
    \item \textbf{RAII:} Explica qué es "Resource Acquisition Is Initialization".
    \item \textbf{Dangling Pointer:} ¿Qué es un puntero colgante?
    \item \textbf{Memory Leak:} Describe cómo se produce una fuga de memoria.
    \item \textbf{Unique vs Shared:} ¿Cuándo usarías un \texttt{shared\_ptr} en lugar de un \texttt{unique\_ptr}?
    \item \textbf{Nullptr:} ¿Por qué usamos \texttt{nullptr} en lugar de NULL?
\end{enumerate}

\subsection{Retos de Desarrollo (10 Ejercicios)}
\begin{enumerate}
    \item **Fábrica:** Crea una función que devuelva un \texttt{unique\_ptr} de entero.
    \item **Intercambio:** Usa punteros para intercambiar el valor de dos variables.
    \item **Arreglo Dinámico:** Crea un array en el Heap usando smart pointers.
    \item **Transferencia:** Intenta copiar un \texttt{unique\_ptr} (fallará) y luego muévelo con \texttt{std::move}.
    \item **Lista Enlazada:** Implementa un nodo que apunte al siguiente usando \texttt{unique\_ptr}.
    \item **Weak Ptr:** Investiga para qué sirve \texttt{std::weak\_ptr}.
    \item **Custom Deleter:** Usa un smart pointer para gestionar un archivo.
    \item **Polimorfismo:** Crea un vector de \texttt{unique\_ptr} a Animales.
    \item **Referencia:** Pasa un smart pointer a una función por referencia.
    \item **Debug:** Crea una clase que imprima mensajes en su constructor y destructor.
\end{enumerate}

\subsection{Revisión de Código}
\begin{codebox}
\begin{lstlisting}
void leak() {
    int* e = new int(10);
    // Error: Salimos sin hacer delete
}

int* obtener() {
    int local = 10;
    return &local; // Error: Retorna direccion de variable muerta
}
\end{lstlisting}
\end{codebox}
\chapter{Herencia y Polimorfismo}

\section{Herencia: Reutilizando Código}
La herencia nos permite crear nuevas clases basadas en clases existentes.

\begin{codebox}
\begin{lstlisting}
// Clase Base
class Personaje {
public:
    std::string nombre;
    void moverse() { std::cout << "Moviendose..."; }
};

// Clase Derivada
class Mago : public Personaje {
public:
    void lanzarHechizo() { std::cout << "Fuego!"; }
};
\end{lstlisting}
\end{codebox}

\section{Polimorfismo y Virtual}
El polimorfismo permite tratar a objetos diferentes como si fueran lo mismo.
Usamos la palabra clave \texttt{virtual}.

\begin{codebox}
\begin{lstlisting}
class Animal {
public:
    virtual void hacerSonido() {
        std::cout << "Sonido generico...\n";
    }
};

class Perro : public Animal {
public:
    void hacerSonido() override {
        std::cout << "Guau!\n";
    }
};
\end{lstlisting}
\end{codebox}

\section{Clases Abstractas}
Son clases que no se pueden instanciar. Tienen al menos una función virtual pura.

\begin{codebox}
\begin{lstlisting}
class Forma {
public:
    virtual double area() = 0; // Metodo puro
};

class Circulo : public Forma {
    double r;
public:
    double area() override { return 3.14 * r * r; }
};
\end{lstlisting}
\end{codebox}

\newpage
\section{Evaluación y Práctica Profesional}

\subsection{Ejercicios}
\begin{enumerate}
    \item Diferencia entre herencia Pública, Privada y Protegida.
    \item ¿Para qué sirve la palabra clave \texttt{override}?
    \item ¿Por qué el Destructor base debe ser \texttt{virtual}?
    \item ¿Qué es una Tabla Virtual (vtable)?
    \item ¿Se puede instanciar una clase abstracta?
    \item **Sistema:** Crea una base \texttt{Empleado} y derivadas \texttt{Gerente} y \texttt{Obrero}.
    \item **Geometría:** Implementa \texttt{Forma} abstracta y clases hijas.
    \item **Vehículos:** Jerarquía \texttt{Vehiculo} con método \texttt{conducir()}.
    \item **Vector Polimórfico:** Crea un \texttt{vector} de punteros inteligentes \texttt{unique\_ptr} a Animal.
    \item **Constructores:** Llama al constructor del padre desde el hijo.
\end{enumerate}

\subsection{Revisión de Código}
\begin{codebox}
\begin{lstlisting}
class Base { 
    ~Base() {} // Error: Debe ser virtual
};

class Derivada : public Base {
    ~Derivada() { /* Limpieza */ }
};

Base* p = new Derivada();
delete p; // Error: No llama al destructor de Hija
\end{lstlisting}
\end{codebox}
\chapter{Manejo de Errores y Excepciones}

\section{El Problema de los Errores}
En el estilo antiguo (C), las funciones devolvían códigos de error. C++ utiliza **Excepciones**. Cuando algo sale mal, la función "lanza" (\textit{throw}) un problema, y alguien más arriba debe "atraparlo" (\textit{catch}).

\section{Sintaxis Básica: try, catch, throw}
\begin{enumerate}
    \item \textbf{try:} Envuelve el código peligroso.
    \item \textbf{throw:} Se ejecuta cuando detectamos el error.
    \item \textbf{catch:} Maneja el error.
\end{enumerate}

\begin{codebox}
\begin{lstlisting}
#include <iostream>
#include <stdexcept>

double dividir(double numerador, double denominador) {
    if (denominador == 0) {
        throw std::runtime_error("Error: Division por cero!");
    }
    return numerador / denominador;
}

int main() {
    try {
        double resultado = dividir(10, 0);
        std::cout << resultado;
    } 
    catch (const std::runtime_error& e) {
        std::cerr << "Ocurrio un problema: " << e.what();
    }
    return 0;
}
\end{lstlisting}
\end{codebox}

\section{Desenrollado de Pila (Stack Unwinding)}
Cuando se lanza una excepción, C++ destruye todas las variables locales hasta encontrar un \texttt{catch}.

\begin{codebox}
\begin{lstlisting}
void funcionPeligrosa() {
    // Si hay error, este unique_ptr se libera solo
    // std::unique_ptr<int> ptr...
    throw std::runtime_error("Fallo critico");
}
\end{lstlisting}
\end{codebox}

\newpage
\section{Evaluación Profesional}

\subsection{Ejercicios}
\begin{enumerate}
    \item **División Segura:** Crea una función que lance \texttt{std::invalid\_argument} si divides por 0.
    \item **Acceso Vectorial:** Intenta acceder al índice 10 de un vector de 3 elementos usando \texttt{.at()}. Captura \texttt{out\_of\_range}.
    \item **Memory Leak Check:** Usa un puntero crudo y lanza una excepción antes del delete. Comprueba la fuga. Luego soluciónalo con \texttt{unique\_ptr}.
    \item **Rethrow (Relanzar):** Captura una excepción, imprime un log y vuelve a lanzarla usando \texttt{throw;} sin argumentos.
    \item **Constructor Fallido:** Crea una clase que lance error en su constructor.
\end{enumerate}

\subsection{Revisión de Código}
\begin{codebox}
\begin{lstlisting}
try { 
    // Codigo...
} 
catch (std::exception e) { 
    // Error: Falta & (Slicing) y const
    std::cout << e.what(); 
}

try { 
    // Codigo...
}
catch (std::exception& e) { } // Generico
catch (std::runtime_error& e) { } // Especifico
// Error: El generico atrapa todo primero. Orden incorrecto.
\end{lstlisting}
\end{codebox}
\chapter{Templates y Programación Genérica}

\section{¿Qué son los Templates?}
Los **Templates** (Plantillas) permiten escribir código que funciona con cualquier tipo de dato. Es la base de la STL.

\section{Funciones Template}
Usamos la palabra clave \texttt{template}.

\begin{codebox}
\begin{lstlisting}
// T representa "Cualquier Tipo"
template <typename T>
T sumar(T a, T b) {
    return a + b;
}

int main() {
    std::cout << sumar(10, 20);      // T es int
    std::cout << sumar(1.5, 2.5);    // T es double
    // sumar(10, 2.5); // Error: Ambiguedad
}
\end{lstlisting}
\end{codebox}

\section{Clases Template}
\begin{codebox}
\begin{lstlisting}
template <typename T>
class Caja {
    T contenido;
public:
    Caja(T valor) : contenido(valor) {}
    T obtener() { return contenido; }
};

int main() {
    Caja<int> cajaInt(100);
    Caja<std::string> cajaTexto("Hola");
}
\end{lstlisting}
\end{codebox}

\newpage
\section{Evaluación Profesional}

\subsection{Ejercicios}
\begin{enumerate}
    \item **Intercambio:** Implementa tu propia función \texttt{mi\_swap<T>}.
    \item **Buscador:** Crea una función que busque un valor \texttt{T} en un vector.
    \item **Calculadora:** Crea una clase \texttt{Calculadora<T>} con métodos sumar y restar.
    \item **Par:** Implementa una clase \texttt{Par<K, V>} para guardar clave y valor.
    \item **Promedio:** Crea una función que reciba un vector de \texttt{T} y devuelva el promedio (cuidado con los tipos).
\end{enumerate}

\subsection{Revisión de Código}
\begin{codebox}
\begin{lstlisting}
template <typename T> 
T max(T a, T b) { return (a>b)?a:b; }

int main() {
    // std::cout << max(10, 5.5); 
    // Error: T no puede ser int y double a la vez.
    // Solucion: max<double>(10, 5.5);
}

template <typename T>
void imprimir(T obj) { std::cout << obj; }

struct Perro {};
// imprimir(Perro()); 
// Error: cout no sabe como imprimir un Perro
\end{lstlisting}
\end{codebox}
\chapter{La STL: Algoritmos y Lambdas}

\section{La Filosofía de la STL}
No reinventes la rueda. Antes de escribir un bucle para buscar u ordenar, revisa la librería \texttt{<algorithm>}.

\section{Algoritmos Comunes}
\begin{codebox}
\begin{lstlisting}
#include <algorithm>
#include <vector>
#include <iostream>

int main() {
    std::vector<int> nums = {5, 2, 9, 1};

    // Ordenar (Sort)
    std::sort(nums.begin(), nums.end()); 

    // Buscar (Find)
    auto it = std::find(nums.begin(), nums.end(), 9);
    
    if (it != nums.end()) {
        std::cout << "Encontrado: " << *it;
    }
    return 0;
}
\end{lstlisting}
\end{codebox}

\section{Lambdas (Funciones Anónimas)}
Son funciones pequeñas escritas "al vuelo".
\texttt{[](params) \{ cuerpo \}}

\begin{codebox}
\begin{lstlisting}
std::vector<int> datos = {1, 2, 3};
int multiplicador = 10;

// [=] captura variables externas por copia
std::for_each(datos.begin(), datos.end(), [=](int n) {
    std::cout << n * multiplicador << " ";
});
\end{lstlisting}
\end{codebox}

\newpage
\section{Evaluación Profesional}

\subsection{Ejercicios}
\begin{enumerate}
    \item **Sort Personalizado:** Ordena un vector de mayor a menor usando una lambda.
    \item **Filtrado:** Usa \texttt{std::copy\_if} para copiar solo números pares.
    \item **Búsqueda Avanzada:** Usa \texttt{std::find\_if} para encontrar el primer número mayor a 100.
    \item **Transformación:** Usa \texttt{std::transform} para convertir strings a mayúsculas.
    \item **Contador:** Cuenta cuántos negativos hay con \texttt{std::count\_if}.
\end{enumerate}

\subsection{Revisión de Código}
\begin{codebox}
\begin{lstlisting}
std::vector<int> v = {1, 2, 3};
// std::sort(v.begin(), v.begin()); 
// Error: Rango vacio o invalido.

auto it = std::find(v.begin(), v.end(), 5);
// std::cout << *it; 
// Error: Si no lo encuentra, it es v.end().
// Dereferenciar end() es comportamiento indefinido.
\end{lstlisting}
\end{codebox}
\chapter{Archivos y Persistencia de Datos}

\section{Introducción a fstream}
Hasta ahora, todos los datos se pierden al cerrar el programa. Para guardarlos permanentemente (persistencia), escribimos en el disco duro.
Usamos la librería \texttt{<fstream>} (File Stream), que maneja el flujo de datos hacia archivos.

\begin{itemize}
    \item \textbf{std::ofstream:} (Output) Para escribir.
    \item \textbf{std::ifstream:} (Input) Para leer.
    \item \textbf{std::fstream:} Para lectura y escritura a la vez.
\end{itemize}

\section{Escribiendo en Archivos}
Funciona casi igual que \texttt{std::cout}.

\begin{codebox}
\begin{lstlisting}
#include <fstream>
#include <iostream>

int main() {
    // 1. Abrir el archivo (se crea si no existe)
    std::ofstream archivo("notas.txt");

    // 2. Verificar si se abrio correctamente
    if (archivo.is_open()) {
        archivo << "Lista de Tareas:\n";
        archivo << "1. Aprender C++\n";
        archivo << "2. Dominar Punteros\n";
        
        // 3. Cerrar (Buena practica, aunque el destructor lo hace)
        archivo.close();
        std::cout << "Guardado con exito.";
    } else {
        std::cerr << "Error al crear el archivo.";
    }
    return 0;
}
\end{lstlisting}
\end{codebox}

\section{Leyendo Archivos}
Leemos línea por línea o palabra por palabra.

\begin{codebox}
\begin{lstlisting}
#include <string>

int main() {
    std::ifstream lectura("notas.txt");
    std::string linea;

    if (lectura.is_open()) {
        // while(getline) lee hasta que se acabe el archivo
        while (std::getline(lectura, linea)) {
            std::cout << "Leido: " << linea << "\n";
        }
        lectura.close();
    } else {
        std::cerr << "No encuentro el archivo.";
    }
    return 0;
}
\end{lstlisting}
\end{codebox}

\section{Modos de Apertura (Flags)}
A veces no queremos borrar lo que había antes. Usamos banderas (flags) para controlar cómo se abre el archivo.

\begin{itemize}
    \item \texttt{std::ios::app} (Append): Escribe al final sin borrar.
    \item \texttt{std::ios::binary}: Modo binario (imágenes, audio).
\end{itemize}

\begin{codebox}
\begin{lstlisting}
// Abre en modo Append (anadir al final)
std::ofstream log("registro.txt", std::ios::app);
log << "Nuevo evento registrado.\n";
\end{lstlisting}
\end{codebox}

\newpage
\section{Evaluación Profesional}

\subsection{Preguntas de Ingeniería}
\begin{enumerate}
    \item \textbf{Buffer:} ¿Qué significa que la escritura en disco tiene "buffer"? ¿Por qué el texto no aparece inmediatamente en el archivo?
    \item \textbf{Rutas:} ¿Cuál es la diferencia entre una ruta relativa ("datos.txt") y una absoluta ("C:/Usuarios/...")?
    \item \textbf{Binario vs Texto:} ¿Por qué no deberías abrir una imagen JPG en modo texto predeterminado?
\end{enumerate}

\subsection{Retos}
\begin{enumerate}
    \item **Logger:** Crea una función `log(mensaje)` que escriba errores en un archivo sin borrar los anteriores.
    \item **Configuración:** Lee un archivo `config.txt` que tenga `puerto=8080` y extrae el número.
    \item **Copia:** Haz un programa que copie un archivo origen a uno destino byte por byte.
\end{enumerate}
\chapter{Semántica de Movimiento}

\section{Optimización con Move}
Copiar objetos grandes es lento. Moverlos (robar sus recursos) es instantáneo.
Usamos \texttt{std::move} y referencias R-value (\texttt{\&\&}).

\begin{codebox}
\begin{lstlisting}
#include <utility>
#include <string>
#include <iostream>

int main() {
    std::string a = "Texto muy largo...";
    
    // Mover: 'b' roba el contenido de 'a'
    std::string b = std::move(a);
    
    std::cout << "A: " << a << "\n"; // A esta vacio
    std::cout << "B: " << b << "\n"; // B tiene el texto
}
\end{lstlisting}
\end{codebox}

\section{Reglas de Oro}
\begin{itemize}
    \item Úsalo con objetos pesados (vectores, strings).
    \item No uses un objeto después de haberlo movido.
    \item Los tipos primitivos (\texttt{int}, \texttt{double}) se copian, no se mueven.
\end{itemize}

\newpage
\section{Evaluación Profesional}

\subsection{Ejercicios}
\begin{enumerate}
    \item **Benchmark:** Mide el tiempo de copiar vs mover un vector gigante.
    \item **Clase Movible:** Implementa un Constructor de Movimiento.
    \item **Unique Ptr:** Intenta meter un \texttt{unique\_ptr} en un vector usando \texttt{push\_back(std::move(ptr))}.
    \item **Swap:** Crea una función swap usando move.
    \item **Retorno:** ¿Por qué no debemos hacer \texttt{return std::move(local)}? (Investiga RVO).
\end{enumerate}
\chapter{Concurrencia y Multithreading}

\section{Hilos (Threads)}
Por defecto, tu código corre en un solo núcleo. Con \texttt{std::thread}, podemos usar todos los núcleos del procesador a la vez.

\begin{codebox}
\begin{lstlisting}
#include <thread>
#include <iostream>

void tarea() {
    std::cout << "Hola desde el hilo secundario\n";
}

int main() {
    // Lanzar hilo
    std::thread t1(tarea);

    // El main sigue ejecutando esto en paralelo...
    std::cout << "Hola desde el Main\n";

    // Esperar a que t1 termine (join)
    t1.join(); 
    return 0;
}
\end{lstlisting}
\end{codebox}

\section{El Peligro: Race Conditions}
Si dos hilos tocan la misma variable a la vez, ocurre un desastre. Necesitamos protección.

\section{Mutex (El Candado)}
\texttt{std::mutex} asegura que solo un hilo pase a la vez.

\begin{codebox}
\begin{lstlisting}
#include <mutex>

int contador = 0;
std::mutex candado;

void incrementar() {
    for(int i=0; i<1000; ++i) {
        // Bloquear acceso
        candado.lock();
        
        contador++; // Zona critica protegida
        
        // Desbloquear
        candado.unlock();
    }
}
\end{lstlisting}
\end{codebox}

\subsection{std::lock\_guard (RAII)}
Es mejor usar \texttt{lock\_guard} para no olvidar desbloquear.

\begin{codebox}
\begin{lstlisting}
void incrementarSeguro() {
    // Se bloquea al crear, se desbloquea solo al salir
    std::lock_guard<std::mutex> guardia(candado);
    contador++;
}
\end{lstlisting}
\end{codebox}

\newpage
\section{Evaluación Profesional}

\subsection{Preguntas de Ingeniería}
\begin{enumerate}
    \item \textbf{Deadlock:} ¿Qué pasa si el Hilo A espera al Hilo B, y el Hilo B espera al Hilo A?
    \item \textbf{Join vs Detach:} ¿Por qué es peligroso usar \texttt{detach()} si el hilo usa variables del main?
    \item \textbf{Race Condition:} Define qué es una Condición de Carrera con tus propias palabras.
\end{enumerate}

\subsection{Retos}
\begin{enumerate}
    \item **Contador Compartido:** Lanza 10 hilos que sumen a una variable global protegida por mutex.
    \item **Productor-Consumidor:** Un hilo agrega datos a una cola, otro los lee.
    \item **Calculadora Paralela:** Divide un array gigante en 4 partes y suma cada parte en un hilo distinto.
\end{enumerate}
\chapter{Proyecto Integrador: Sistema de Biblioteca}

\section{El Objetivo}
Vamos a construir un software completo que gestione una biblioteca. El sistema debe permitir:
\begin{itemize}
    \item Agregar libros nuevos.
    \item Listar todos los libros.
    \item Guardar la base de datos en un archivo.
    \item Cargar la base de datos al iniciar.
\end{itemize}

\section{Diseño de la Clase Libro}
Usaremos una clase simple para representar cada libro.

\begin{codebox}
\begin{lstlisting}
#include <iostream>
#include <vector>
#include <string>
#include <fstream>
#include <memory> // Para smart pointers

class Libro {
public:
    std::string titulo;
    std::string autor;
    int anio;

    Libro(std::string t, std::string a, int y) 
        : titulo(t), autor(a), anio(y) {}

    void mostrar() const {
        std::cout << "Titulo: " << titulo 
                  << " | Autor: " << autor 
                  << " (" << anio << ")\n";
    }
};
\end{lstlisting}
\end{codebox}

\section{El Gestor (La Lógica)}
Esta clase manejará la lista de libros usando un vector de punteros inteligentes.

\begin{codebox}
\begin{lstlisting}
class Biblioteca {
private:
    // Vector de punteros unicos a Libros
    std::vector<std::unique_ptr<Libro>> estanteria;

public:
    void agregarLibro() {
        std::string t, a;
        int y;
        
        std::cout << "Ingrese Titulo: ";
        std::cin.ignore(); 
        std::getline(std::cin, t);
        
        std::cout << "Ingrese Autor: ";
        std::getline(std::cin, a);
        
        std::cout << "Ingrese Anio: ";
        std::cin >> y;

        // Creamos el libro directo en el vector
        estanteria.push_back(std::make_unique<Libro>(t, a, y));
        std::cout << "Libro agregado con exito!\n";
    }

    void listarLibros() {
        if (estanteria.empty()) {
            std::cout << "La biblioteca esta vacia.\n";
            return;
        }
        
        std::cout << "\n--- Catalogo ---\n";
        for (const auto& libro : estanteria) {
            libro->mostrar();
        }
        std::cout << "----------------\n";
    }

    void guardarEnDisco() {
        std::ofstream archivo("biblioteca.txt");
        if (archivo.is_open()) {
            for (const auto& libro : estanteria) {
                // Formato CSV simple: Titulo|Autor|Anio
                archivo << libro->titulo << "|" 
                        << libro->autor << "|" 
                        << libro->anio << "\n";
            }
            std::cout << "Datos guardados en biblioteca.txt\n";
        } else {
            std::cerr << "Error al guardar archivo.\n";
        }
    }
};
\end{lstlisting}
\end{codebox}

\section{El Menú Principal}
Finalmente, conectamos todo en el main.

\begin{codebox}
\begin{lstlisting}
int main() {
    Biblioteca biblio;
    int opcion = 0;

    while (opcion != 4) {
        std::cout << "\n1. Agregar Libro\n";
        std::cout << "2. Listar Libros\n";
        std::cout << "3. Guardar y Salir\n";
        std::cout << "Seleccione: ";
        std::cin >> opcion;

        switch (opcion) {
            case 1: biblio.agregarLibro(); break;
            case 2: biblio.listarLibros(); break;
            case 3: 
                biblio.guardarEnDisco(); 
                opcion = 4; // Salir
                break;
            default: std::cout << "Opcion invalida.\n";
        }
    }
    return 0;
}
\end{lstlisting}
\end{codebox}

\section{Reto Final}
Tu misión para obtener el título de Desarrollador C++ Junior:
\begin{enumerate}
    \item Implementa la función \texttt{cargarDesdeDisco()} para que al abrir el programa, lea el archivo "biblioteca.txt" y reconstruya los objetos Libro en el vector. (Pista: usa \texttt{getline} con delimitador '|').
    \item Añade una opción para buscar un libro por título.
\end{enumerate}

% =========================================
%             CONCLUSIONES
% =========================================
\chapter{Conclusiones y Futuro}

\section{El Camino Recorrido}
Hemos completado un viaje extenso desde los fundamentos más básicos hasta conceptos avanzados de ingeniería de software.
Comenzamos con un simple "Hola Mundo" y terminamos construyendo un sistema de gestión de biblioteca con persistencia de datos, gestión de memoria moderna y arquitectura orientada a objetos.

\section{Lo que has logrado}
Al finalizar este manual, ahora posees conocimientos sobre:
\begin{itemize}
    \item \textbf{C++ Moderno:} El uso de \texttt{auto}, lambdas y la biblioteca estándar (STL).
    \item \textbf{Gestión de Memoria:} El dominio de \texttt{unique\_ptr} y \texttt{shared\_ptr} para evitar fugas.
    \item \textbf{Arquitectura:} El diseño de clases robustas y el uso de herencia y polimorfismo.
    \item \textbf{Optimización:} Conceptos avanzados como la Semántica de Movimiento.
\end{itemize}

\section{Siguientes Pasos}
El aprendizaje de C++ nunca termina. Para continuar tu carrera profesional, te recomiendo explorar:
\begin{enumerate}
    \item \textbf{Patrones de Diseño:} Singleton, Factory, Observer, etc.
    \item \textbf{Frameworks Gráficos:} Qt o wxWidgets para crear interfaces visuales.
    \item \textbf{Redes:} Programación de Sockets para aplicaciones conectadas a internet.
    \item \textbf{Bases de Datos:} Conectar C++ con SQL.
\end{enumerate}

\vspace{2cm}
\begin{center}
    \textit{"El software se come al mundo, y C++ es sus dientes."}
\end{center}

\end{document}